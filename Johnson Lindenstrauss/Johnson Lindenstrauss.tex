% --------------------------------------------------------------
% This is all preamble stuff that you don't have to worry about.
% Head down to where it says ``Start here"
% --------------------------------------------------------------
 
\documentclass[12pt]{article}
 

 
\usepackage[margin=1in]{geometry} 
\usepackage{amsmath,amsthm,amssymb}
\usepackage{graphicx}
\usepackage{tabto}

% \newtheorem{problem}{Problem}
% \newtheorem{solution}{Solution}
 
\newcommand{\N}{\mathbb{N}}
\newcommand{\Z}{\mathbb{Z}}

\newenvironment{solution}[2][Solution]{\begin{trivlist}
\item[\hskip \labelsep {\bfseries #1}\hskip \labelsep {\bfseries #2.}]}{\end{trivlist}}
 
\newenvironment{problem}[2][Problem]{\begin{trivlist}
\item[\hskip \labelsep {\bfseries #1}\hskip \labelsep {\bfseries #2.}]}{\end{trivlist}}
 
\newenvironment{theorem}[2][Theorem]{\begin{trivlist}
\item[\hskip \labelsep {\bfseries #1}\hskip \labelsep {\bfseries #2.}]}{\end{trivlist}}
\newenvironment{claim}[2][Claim]{\begin{trivlist}
		\item[\hskip \labelsep {\bfseries #1}\hskip \labelsep {\bfseries #2.}]}{\end{trivlist}}
\newenvironment{lemma}[2][Lemma]{\begin{trivlist}
\item[\hskip \labelsep {\bfseries #1}\hskip \labelsep {\bfseries #2.}]}{\end{trivlist}}
\newenvironment{exercise}[2][Exercise]{\begin{trivlist}
\item[\hskip \labelsep {\bfseries #1}\hskip \labelsep {\bfseries #2.}]}{\end{trivlist}}
\newenvironment{reflection}[2][Reflection]{\begin{trivlist}
\item[\hskip \labelsep {\bfseries #1}\hskip \labelsep {\bfseries #2.}]}{\end{trivlist}}
\newenvironment{proposition}[2][Proposition]{\begin{trivlist}
\item[\hskip \labelsep {\bfseries #1}\hskip \labelsep {\bfseries #2.}]}{\end{trivlist}}
\newenvironment{corollary}[2][Corollary]{\begin{trivlist}
\item[\hskip \labelsep {\bfseries #1}\hskip \labelsep {\bfseries #2.}]}{\end{trivlist}}
\newenvironment{definition}[2][Definition]{\begin{trivlist}
		\item[\hskip \labelsep {\bfseries #1}\hskip \labelsep {\bfseries #2.}]}{\end{trivlist}}


\newenvironment{example}[2][Example]{\begin{trivlist}
		\item[\hskip \labelsep {\bfseries #1}\hskip \labelsep {\bfseries #2.}]}{\end{trivlist}}
	
\newenvironment{algorithm}[2][Algorithm]{\begin{trivlist}
		\item[\hskip \labelsep {\bfseries #1}\hskip \labelsep {\bfseries #2.}]}{\end{trivlist}}
 
\begin{document}
 
% --------------------------------------------------------------
%                         Start here
% --------------------------------------------------------------
 
%\renewcommand{\qedsymbol}{\filledbox}
 
\title{Johnson-Lindenstrauss, Compress Sensing}
\author{Zhiwei Zhang}
 
\maketitle

\section{Johnson-Lindenstrauss (JL)}
\subsection{Formal Theorem}
For a set of $n$ points $x _ { 1 } , \ldots , x _ { n } \in \mathbb { R } ^ { d }$, then for a random $k = \frac { c \log n } { \varepsilon ^ { 2 } }$ dimensional subspace, we project these points to this subspace. \\
\newline 
JL states that with probability $1 - \frac { 1 } { n ^ { c - 2 } }$, we have:
$$
( 1 - \varepsilon ) \sqrt { \frac { k } { d } } \left| x _ { i } - x _ { j } \right| \leq \left| y _ { i } - y _ { j } \right| \leq ( 1 + \varepsilon ) \sqrt { \frac { k } { d } } \left| x _ { i } - x _ { j } \right|
$$
Furthermore, $y_i$ is the projection of $x_i$ on to the subspace.
\subsection{Intuitive Interpretation}
$|x_i - x_j|$ is the original distance between two points, and similar for  $|y_i - y_j|$. Therefore the inequality is stating that after projecting to the subspace of dimension $k$, with high probability that the pariwise distances are scaled by $\sqrt{\frac{k}{d}}$ within factor of 1$\pm \varepsilon$

\section{Projection Method}
We select a set of $k$ orthogonal unit vectors $$v_1, \cdots, v_k$$ as the basis of the subspace, and project each vector on to them. \\
\newline
For a vector $x \in x _ { 1 } , \ldots , x _ { n }$, we look at its coordinate vector $y _ { 1 } , \ldots , x _ { k }$ in the projection onto the subspace, where $y_i = x \cdot v_i$\\
\newline
Now, consider a rotation $U$ that transform $v_i$ to $e_i$ (elementary vector where the $i$-th entry is 1 and all others are 0). Let $z = Ux$, therefore we have $y_i$ be the $i$-th coordinate of $z$ (we basically transform the coordinate vector to the standard basis).

\section{"Expected Length"}
Notice that $z$ is still a unit vector, therefore:
$$
\sum _ { i \in [ d ] } z _ { i } ^ { 2 } = 1 .
$$
After the projection, we only take the first $k$ coordinates:
$$
E \left[ \sum _ { i \in [ k ] } z _ { i } ^ { 2 } \right] = \frac { k } { d } . \text { Linearity of Expectation. }
$$
(Terefore the length is roughly expected to be $\sqrt{\frac{k}{d}}$, but it's not yet rigorously proven yet).


%However, since we can rotate these $k$ vectors to $e_1, \cdots, e_k$ without changing the relative distances, we can do that for the projections as well.\\
%\newline

%Suppose that rotate from $v_i$ to $e_i$ is $U$, 



\end{document}








