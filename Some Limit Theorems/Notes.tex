% --------------------------------------------------------------
% This is all preamble stuff that you don't have to worry about.
% Head down to where it says ``Start here"
% --------------------------------------------------------------
 
\documentclass[12pt]{article}
 
\usepackage[utf8]{inputenc}
\usepackage[english]{babel}

 
\usepackage[margin=1in]{geometry} 
\usepackage{amsmath,amsthm,amssymb}
\usepackage{graphicx}
\usepackage{tabto}

% \newtheorem{problem}{Problem}
% \newtheorem{solution}{Solution}
 
\newcommand{\N}{\mathbb{N}}
\newcommand{\Z}{\mathbb{Z}}
\newtheorem{theorem}{Theorem}[section]
\newtheorem{corollary}{Corollary}[theorem]
\newtheorem{lemma}[theorem]{Lemma}
\newtheorem{definition}[theorem]{Definition}

\newenvironment{solution}[2][Solution]{\begin{trivlist}
\item[\hskip \labelsep {\bfseries #1}\hskip \labelsep {\bfseries #2.}]}{\end{trivlist}}
 
\newenvironment{problem}[2][Problem]{\begin{trivlist}
\item[\hskip \labelsep {\bfseries #1}\hskip \labelsep {\bfseries #2.}]}{\end{trivlist}}
 
%\newenvironment{theorem}[2][Theorem]{\begin{trivlist}
%\item[\hskip \labelsep {\bfseries #1}\hskip \labelsep {\bfseries #2.}]}{\end{trivlist}}
\newenvironment{claim}[2][Claim]{\begin{trivlist}
		\item[\hskip \labelsep {\bfseries #1}\hskip \labelsep {\bfseries #2.}]}{\end{trivlist}}
\newenvironment{exercise}[2][Exercise]{\begin{trivlist}
\item[\hskip \labelsep {\bfseries #1}\hskip \labelsep {\bfseries #2.}]}{\end{trivlist}}
\newenvironment{reflection}[2][Reflection]{\begin{trivlist}
\item[\hskip \labelsep {\bfseries #1}\hskip \labelsep {\bfseries #2.}]}{\end{trivlist}}
\newenvironment{proposition}[2][Proposition]{\begin{trivlist}
\item[\hskip \labelsep {\bfseries #1}\hskip \labelsep {\bfseries #2.}]}{\end{trivlist}}



\newenvironment{example}[2][Example]{\begin{trivlist}
		\item[\hskip \labelsep {\bfseries #1}\hskip \labelsep {\bfseries #2.}]}{\end{trivlist}}
	
\newenvironment{algorithm}[2][Algorithm]{\begin{trivlist}
		\item[\hskip \labelsep {\bfseries #1}\hskip \labelsep {\bfseries #2.}]}{\end{trivlist}}
 
\begin{document}
 
% --------------------------------------------------------------
%                         Start here
% --------------------------------------------------------------
 
%\renewcommand{\qedsymbol}{\filledbox}
 
\title{Real Analysis Theorems}
\author{Zhiwei Zhang}
 
\maketitle

\section{Introduction}
\begin{theorem}[Complete Axiom]
	Every nonempty subset $S$ of $\mathbb { R }$ that is bounded above has a least upper
	bound. In other words, supS exists and is a real number.
\end{theorem}

\section{Monotone Sequence and Cauchy Sequences}
\begin{theorem}
	All bounded monotone sequences converge.
\end{theorem}
\begin{definition}
	A sequence $\left( s _ { n } \right)$ of real numbers is called a Cauchy sequence if
	for each $\epsilon > 0$ there exists a number $N$ such that
	$m , n > N$ implies $\left| s _ { n } - s _ { m } \right| < \epsilon$
\end{definition}
\begin{lemma}
	Convergent sequences are Cauchy sequences.
\end{lemma}

\begin{theorem}
	A sequence is a convergent sequence if and only if it is a Cauchy sequence.
\end{theorem}
\begin{proof}[sketch proof]$$
	\left| s _ { n } - s _ { m } \right| = \left| s _ { n } - s + s - s _ { m } \right| \leq \left| s _ { n } - s \right| + \left| s - s _ { m } \right|
	$$
\end{proof}

\section{Subsequences}
\begin{theorem}
	Let $(s_n)$ be a sequence. If $t$ is in $\mathbb { R } ,$ then there is a subsequence of $\left( s _ { n } \right)$ converging to $t$
	if and only if the set $\left\{ n \in \mathbb { N } : \left| s _ { n } - t \right| < \epsilon \right\}$ is infinite for all
	$\epsilon > 0$
\end{theorem}

\begin{theorem}
	If the sequence $\left( s _ { n } \right)$ converges, then every subsequence converges to
	the same limit.
\end{theorem}

\begin{theorem}
	Every sequence $(sn)$ has a monotonic subsequence.
\end{theorem}

\begin{theorem}[Bolzano-Weierstrass Theorem]
	Every bounded sequence has a convergent subsequence.
\end{theorem} 
\begin{theorem}
	Let $\left( s _ { n } \right)$ be any sequence. There exists a monotonic subsequence
	whose limit is limsup $s _ { n } ,$ and there exists a monotonic subsequence
	whose limit is liminf $s _ { n } .$
\end{theorem}

\begin{theorem}
	Let $\left( s _ { n } \right)$ be any sequence in $\mathbb { R } ,$ and let $S$ denote the set of
	subsequential limits of $\left( s _ { n } \right) .$
	\begin{itemize}
		\item $S$ is nonempty
		\item $\sup S = \limsup s _ { n }$ and inf $S = \liminf s _ { n }$
		\item $\lim s _ { n }$ exists if and only if $S$ has exactly one element, namely $\lim s _ { n } .$
	\end{itemize}
\end{theorem}

\begin{theorem}
	Let $S$ denote the set of subsequential limits of a sequence $\left( s _ { n } \right) .$ Sup-
	pose $\left( t _ { n } \right)$ is a sequence in $S \cap$ and that $t = \lim t _ { n } .$ Then $t$ belongs
	to $S $ (which means $S$ is \textbf{closed}).
\end{theorem}
\section{Limsup and Liminf}
\begin{theorem}
	If $\left( s _ { n } \right)$ converges to a positive real number $s$ and $\left( t _ { n } \right)$ is any sequence,
	then $$
	\limsup s _ { n } t _ { n } = s \cdot \limsup t _ { n }
	$$ Here we allow the conventions $s \cdot ( + \infty ) = + \infty$ and $s \cdot ( - \infty ) = - \infty$
	for $s > 0 .$
\end{theorem}
\begin{theorem}
	Let $\left( s _ { n } \right)$ be any sequence of nonzero real numbers. Then we have
	$$\quad \liminf \left| \frac { s _ { n + 1 } } { s _ { n } } \right| \leq \liminf \left| s _ { n } \right| ^ { 1 / n } \leq \limsup \left| s _ { n } \right| ^ { 1 / n } \leq \limsup \left| \frac { s _ { n + 1 } } { s _ { n } } \right|$$
\end{theorem}
\begin{corollary}
	If $\lim \left| \frac { s _ { n + 1 } } { s _ { n } } \right|$ exists $[$ and equals $L ] ,$ then $\lim \left| s _ { n } \right| ^ { 1 / n } $ exists $[$ and
	equals $L ] .$
\end{corollary}
\section{Series}
\begin{theorem}[Ratio Test]
A series $\sum a_{n}$ of nonzero terms
\begin{enumerate}
	\item Converges absolutely if limsup $\left| a _ { n + 1 } / a _ { n } \right| < 1$
	\item Diverges if liminf $\left| a _ { n + 1 } / a _ { n } \right| > 1$
	\item Otherwise liminf $\left| a _ { n + 1 } / a _ { n } \right| \leq 1 \leq \limsup \left| a _ { n + 1 } / a _ { n } \right|$ and the test gives no information.
\end{enumerate}
\end{theorem}

\begin{theorem}[Root Test]
	Let $\sum a _ { n }$ be a series and let $\alpha = \limsup \left| a _ { n } \right| ^ { 1 / n } .$ The series $\sum a _ { n }$ 
	\begin{enumerate}
		\item converges absolutely if $\alpha < 1$
		\item diverges if $\alpha > 1$
		\item ambiguous if $\alpha = 1$
	\end{enumerate}
\end{theorem}

\begin{theorem}
	If the terms $a _ { n }$ are nonzero and if $\lim \left| a _ { n + 1 } / a _ { n } \right| = 1 ,$ then $\alpha =$
	$\limsup _ { \left| a _ { n } \right| ^ { 1 / n } } = 1$ by Corollary $4.2.1 ,$ so neither the Ratio Test nor
	the Root Test gives information concerning the convergence of $\sum a _ { n }$
\end{theorem}

\section{Alternating Series and Integral Test}
\textbf{Basic Idea is that to compare a series with an integral to test for divergence or convergence.}

\begin{theorem}[Alternating Series Theorem]
	If $a _ { 1 } \geq a _ { 2 } \geq \cdots \geq a _ { n } \geq \cdots \geq 0$ and $\lim a _ { n } = 0 ,$ then the al-
	ternating series$\sum ( - 1 ) ^ { n + 1 } a _ { n }$ converges. Moreover, the partial sums
	$s _ { n } = \sum _ { k = 1 } ^ { n } ( - 1 ) ^ { k + 1 } a _ { k }$ satisfy $\left| s - s _ { n } \right| \leq a _ { n }$ for all $n .$
\end{theorem}

\section{Uniform Continuity}
\begin{theorem}
	Pass. ...
\end{theorem}
\section{Power Series}
Given power series $\sum _ { n = 0 } ^ { \infty } a _ { n } x ^ { n }$
\begin{theorem}
Given any $(a_n)$, one of the following holds true:
\begin{enumerate}
	\item The power series converges for all $x \in \mathbb{R}$
	\item The power series converges only for $x = 0$
	\item The power series converges for all $x$ in some bounded interval centered at $0 ;$ the intervalmay be open, half-open or closed.
\end{enumerate}
\end{theorem}

\begin{theorem}
	Let $$
	\beta = \limsup \left| a _ { n } \right| ^ { 1 / n } \quad \text { and } \quad R = \frac { 1 } { \beta }
	$$
	Then
	\begin{enumerate}
		\item The power series converges for  $|x| < R$
		\item The power series diverges for $|x| > R$
	\end{enumerate}
	
	Also notice that $\lim \left| \frac { a _ { n + 1 } } { a _ { n } } \right| = \beta$, therefore most of the time we will use $\lim \left| \frac { a _ { n + 1 } } { a _ { n } } \right|$ as it's easier to compute thatn $\beta$.
\end{theorem}
\section{More on Uniform Convergence}
\begin{theorem}
	Let $\left( f _ { n } \right)$ be a sequence of continuous functions on $[ a , b ] ,$ and suppose
	$f _ { n } \rightarrow f$ uniformly on $[ a , b ] .$ Then
	$$\lim _ { n \rightarrow \infty } \int _ { a } ^ { b } f _ { n } ( x ) d x = \int _ { a } ^ { b } f ( x ) d x$$
\end{theorem}
\begin{definition}
	A sequence $\left( f _ { n } \right)$ of functions defined on a set $S \subseteq \mathbb { R }$ is uniformly
	Cauchy on $S$ if
	\begin{center}
			$\quad$ for each $\epsilon > 0$ there exists a number $N$ such that\\
		$\quad \left| f _ { n } ( x ) - f _ { m } ( x ) \right| < \epsilon$ for all $x \in S$ and all $m , n > N$
	\end{center}

\end{definition}
\begin{theorem}
	Let $\left( f _ { n } \right)$ be a sequence of functions defined and uniformly Cauchy on
	a set $S \subseteq \mathbb { R } .$ Then there exists a function $f$ on $S$ such that $f _ { n } \rightarrow f$
	uniformly on $S .$
\end{theorem}

\begin{theorem}
	Consider a series $\sum _ { k = 0 } ^ { \infty } g _ { k }$ of functions on a set $S \subseteq \mathbb { R } .$ Suppose
	each $g _ { k }$ is continuous on $S$ and the series converges uniformly on $S$ .
	Then the series $\sum _ { k = 0 } ^ { \infty } g _ { k }$ represents a continuous function on $S .$
\end{theorem}
\begin{theorem}
	If a series $\sum _ { k = 0 } ^ { \infty } g _ { k }$ of functions satisfies the Cauchy criterion
	uniformly on a set $S$ , then the series converges uniformly on $S$ .
\end{theorem}
\begin{theorem}
	Let $\left( M _ { k } \right)$ be a sequence of nonnegative real numbers where $\sum M _ { k } <$
	$\infty .$ If $\left| g _ { k } ( x ) \right| \leq M _ { k }$ for all $x$ in a set $S ,$ then $\sum g _ { k }$ converges
	uniformly on $S .$
\end{theorem}
\begin{theorem}
	Show that if the series $\sum g _ { n }$ converges uniformly on a set $S ,$ then
	$\lim _ { n \rightarrow \infty } \sup \left\{ \left| g _ { n } ( x ) \right| : x \in S \right\} = 0$
\end{theorem}

\section{Differentiation and Integration of Power Series}
\begin{theorem}
	Let $\sum _ { n = 0 } ^ { \infty } a _ { n } x ^ { n }$ be a power series with radius of convergence $R > 0$
	$[$possibly$R = + \infty ] .$ If $0 < R _ { 1 } < R ,$ then the power series converges
	uniformly on $\left[ - R _ { 1 } , R _ { 1 } \right]$ to a continuous function.
\end{theorem}
\begin{lemma}
	If the power series $\sum _ { n = 0 } ^ { \infty } a _ { n } x ^ { n }$ has radius of convergence $R ,$ then the
	power series
	$$\sum _ { n = 1 } ^ { \infty } n a _ { n } x ^ { n - 1 } \quad$$ and $$\sum _ { n = 0 } ^ { \infty } \frac { a _ { n } } { n + 1 } x ^ { n + 1 }$$ also have radius of convergence $R .$
\end{lemma}

\begin{theorem}[Abel’s Theorem]
	Let $f ( x ) = \sum _ { n = 0 } ^ { \infty } a _ { n } x ^ { n }$ be a power series with finite positive radius of
	convergence$R .$ If the series converges at $x = R ,$ then $f$ is continuous
	at $x = R .$ If the series converges at $x = - R ,$ then $f$ is continuous
	at $x = - R .$
\end{theorem}

\section{Basic Properties of the Derivative}
\begin{theorem}
	Differentiability implies continuity.
\end{theorem}

\section{Mean Value Theorem}
\begin{theorem}
	Let $f$ be a continuous function on [a,b] that is differentiable on $( a , b ) .$
	Then there exists [at least one] $x$ in $( a , b )$ such that $$
	f ^ { \prime } ( x ) = \frac { f ( b ) - f ( a ) } { b - a }
	$$
\end{theorem}
\begin{theorem}[Intermediate Value Theorem for Derivatives]
	Let $f$ be a differentiable function on $( a , b ) .$ If $a < x _ { 1 } < x _ { 2 } < b ,$ and if $c$ lies between $f ^ { \prime } \left( x _ { 1 } \right)$ and $f ^ { \prime } \left( x _ { 2 } \right) ,$ there exists [at least one] $x$ in
	$\left( x _ { 1 } , x _ { 2 } \right)$ such that $f ^ { \prime } ( x ) = c$
\end{theorem}

\begin{theorem}
	Let $f$ be a one-to-one continuous function on an open intervalI, and
	let $J = f ( I ) .$ If $f$ is differentiable at $x _ { 0 } \in I$ and if $f ^ { \prime } \left( x _ { 0 } \right) \neq 0 ,$ then
	$f ^ { - 1 }$ is differentiable at $y _ { 0 } = f \left( x _ { 0 } \right)$ and 
	$$
	\left( f ^ { - 1 } \right) ^ { \prime } \left( y _ { 0 } \right) = \frac { 1 } { f ^ { \prime } \left( x _ { 0 } \right) }
	$$
\end{theorem}

\begin{corollary}
	Let $f$ be a differentiable function on $( a , b )$ such that $f ^ { \prime } ( x ) = 0$ for all
	$x \in ( a , b ) .$ Then $f$ is a constant function on $( a , b ) .$
\end{corollary}

\begin{corollary}
	Let $f$ and $g$ be differentiable functions on $( a , b )$ such that $f ^ { \prime } = g ^ { \prime }$ on
	$( a , b ) .$ Then there exists a constant c such that $f ( x ) = g ( x ) + c$ for
	all $x \in ( a , b ) .$
\end{corollary}

\begin{theorem}[IVT for derivatives]
	Let $f$ be a differentiable function on $( a , b ) .$ If $a < x _ { 1 } < x _ { 2 } < b ,$ and
	if c lies between $f ^ { \prime } \left( x _ { 1 } \right)$ and $f ^ { \prime } \left( x _ { 2 } \right) ,$ there exists [at least one] $x$ in
	$\left( x _ { 1 } , x _ { 2 } \right)$ such that $f ^ { \prime } ( x ) = c$
\end{theorem}

\begin{theorem}[Rolle’s Theorem]
	Let $f$ be a continuous function on [a,b] that is differentiable on (a,b)
	and satisfies $f ( a ) = f ( b ) .$ There exists [at least one] $x$ in $( a , b )$ such
	that $f ^ { \prime } ( x ) = 0$
\end{theorem}

\section{Taylor's Theorem}
\begin{definition}
	Taylor's Theorem: $$
	\sum _ { k = 0 } ^ { \infty } \frac { f ^ { ( k ) } ( c ) } { k ! } ( x - c ) ^ { k }
	$$
	Remainder: $$
	R _ { n } ( x ) = f ( x ) - \sum _ { k = 0 } ^ { n - 1 } \frac { f ^ { ( k ) } ( c ) } { k ! } ( x - c ) ^ { k }
	$$
\end{definition}
\begin{theorem}[Taylor's Theorem]
	Let $f$ be defined on (a,b) where $a < c < b ;$ here we allow $a = - \infty$
	or $b = \infty .$ Suppose the nth derivative $f ^ { ( n ) }$ exists on $( a , b ) .$ Then for
	each $x \neq c$ in $( a , b )$ there is some $y$ between $c$ and $x$ such that $$
	R _ { n } ( x ) = \frac { f ^ { ( n ) } ( y ) } { n ! } ( x - c ) ^ { n }
	$$
\end{theorem}
\begin{corollary}
	Let $f$ be defined on $( a , b )$ where $a < c < b .$ If all the derivatives $f ^ { ( n ) }$
	exist on $( a , b )$ and are bounded by a single constant $C ,$ then
	$$
	\lim _ { n \rightarrow \infty } R _ { n } ( x ) = 0 \quad \text { for all } \quad x \in ( a , b )
	$$
\end{corollary}

\begin{theorem}[Taylor's Theorem (another one)]
	Let $f$ be defined on $( a , b )$ where $a < c < b ,$ and suppose the nth
	derivative $f ^ { ( n ) }$ exists and is continuous on $( a , b ) .$ Then for $x \in ( a , b )$ we have $$
	R _ { n } ( x ) = \int _ { c } ^ { x } \frac { ( x - t ) ^ { n - 1 } } { ( n - 1 ) ! } f ^ { ( n ) } ( t ) d t
	$$
\end{theorem}

\begin{corollary}
	If $f$ is as in Theorem 31.5, then for each $x$ in $( a , b )$ different from $c$
	there is some $y$ between $c$ and $x$ such that $$
	R _ { n } ( x ) = ( x - c ) \cdot \frac { ( x - y ) ^ { n - 1 } } { ( n - 1 ) ! } f ^ { ( n ) } ( y )
	$$ This form of $R _ { n }$ is known as Cauchy's form of the remainder.
\end{corollary}

\begin{theorem}[Binomial Series Theorem]
	If $\alpha \in \mathbb { R }$ and $| x | < 1 ,$ then $$
	( 1 + x ) ^ { \alpha } = 1 + \sum _ { k = 1 } ^ { \infty } \frac { \alpha ( \alpha - 1 ) \cdots ( \alpha - k + 1 ) } { k ! } x ^ { k }
	$$
\end{theorem}

\begin{theorem}[Newton's Method]
	Newton's method for finding an approximate solution to $f ( x ) = 0$ is
	to begin with a reasonable initial guess $x _ { 0 }$ and then compute
	$$
	x _ { n } = x _ { n - 1 } - \frac { f \left( x _ { n - 1 } \right) } { f ^ { \prime } \left( x _ { n - 1 } \right) } \quad \text { for } \quad n \geq 1
	$$
\end{theorem}


\begin{theorem}[Secant Method]
	A similar approach to approximating solutions of $f ( x ) = 0$ is to start
	with two reasonable guesses $x _ { 0 }$ and $x _ { 1 }$ and then compute
	$$
	x _ { n } = x _ { n - 1 } - \frac { f \left( x _ { n - 1 } \right) \left( x _ { n - 2 } - x _ { n - 1 } \right) } { f \left( x _ { n - 2 } \right) - f \left( x _ { n - 1 } \right) } \quad \text { for } \quad n \geq 2
	$$
\end{theorem}
\section{The Riemann Integral}
\begin{definition}
	$$
	M ( f , S ) = \sup \{ f ( x ) : x \in S \} \quad \text { and } \quad m ( f , S ) = \inf \{ f ( x ) : x \in S \}
	$$
	A partition of $[ a , b ]$ is any finite ordered subset $P$ having the form $$
	P = \left\{ a = t _ { 0 } < t _ { 1 } < \cdots < t _ { n } = b \right\}
	$$
	The upper Darboux sum $U ( f , P )$ of $f$ with respect to $P$ is the sum $$
	U ( f , P ) = \sum _ { k = 1 } ^ { n } M \left( f , \left[ t _ { k - 1 } , t _ { k } \right] \right) \cdot \left( t _ { k } - t _ { k - 1 } \right)
	$$ and the lower Darboux sum $L ( f , P )$ is $$
	L ( f , P ) = \sum _ { k = 1 } ^ { n } m \left( f , \left[ t _ { k - 1 } , t _ { k } \right] \right) \cdot \left( t _ { k } - t _ { k - 1 } \right)
	$$
\end{definition}

\begin{lemma}
	Let $f$ be a bounded function on $[ a , b ] .$ If $P$ and $Q$ are partitions of
	$[ a , b ]$ and $P \subseteq Q ,$ then $$
	L ( f , P ) \leq L ( f , Q ) \leq U ( f , Q ) \leq U ( f , P )
	$$.
\end{lemma}
\begin{lemma}
	If $f$ is a bounded function on $[ a , b ] ,$ and if $P$ and $Q$ are partitions of
	$[ a , b ] ,$ then $L ( f , P ) \leq U ( f , Q )$
\end{lemma}
\begin{theorem}
	A bounded function $f$ on $[ a , b ]$ is integrable if and only if for each
	$\epsilon > 0$ there exists a partition $P$ of $[ a , b ]$ such that $$
	U ( f , P ) - L ( f , P ) < \epsilon
	$$
\end{theorem}
\begin{definition}
	The mesh of a partition $P$ is the maximum length of the subintervals
	comprising $P .$ Thus if $$
	P = \left\{ a = t _ { 0 } < t _ { 1 } < \cdots < t _ { n } = b \right\}
	$$ then $$
	\operatorname { mesh } ( P ) = \max \left\{ t _ { k } - t _ { k - 1 } : k = 1,2 , \ldots , n \right\}
	$$
\end{definition}
\begin{theorem}
	$A$ bounded function $f$ on $[ a , b ]$ is integrable if and only if for each
	$\epsilon > 0$ there exists $a \delta > 0$ such that $$
	\operatorname { mesh } ( P ) < \delta \quad \text { implies } \quad U ( f , P ) - L ( f , P ) < \epsilon
	$$ for all partitions $P$ of $[ a , b ]$.
\end{theorem}

\begin{definition}
	The function $f$ is Riemann integrable on $[ a , b ]$ if there exists a
	number $r$ with the following property. For each $\epsilon > 0$ there exists
	$\delta > 0$ such that $$
	| S - r | < \epsilon
	$$ for every Riemann sum $S$ of $f$ associated with a partition $P$ having
	$\operatorname { mesh } ( P ) < \delta .$ 
\end{definition}
\begin{theorem}
	A bounded function $f$ on $[ a , b ]$ is Riemann integrable if and only if
	it is $[$ Darboux] integrable, in which case the values of the integrals
	agree.
\end{theorem}
\begin{corollary}
	Let $f$ be a bounded Riemann integrable function on [a,b]. Suppose
	$\left( S _ { n } \right)$ is a sequence of Riemann sums, with corresponding partitions $P _ { n } ,$ satisfying lim $_ { n } \operatorname { mesh } \left( P _ { n } \right) = 0 .$ Then the sequence $\left( S _ { n } \right)$ converges
	to $\int _ { a } ^ { b } f .$
\end{corollary}
\begin{theorem}
	Every monotonic function $f$ on $[ a , b ]$ is integrable.
\end{theorem}
\begin{theorem}
	Every continuous function $f$ on $[ a , b ]$ is integrable.
\end{theorem}
\begin{theorem}
	If $f$ is integrable on $[ a , b ] ,$ then $| f |$ is integrable on $[ a , b ]$ and $$
	\left| \int _ { a } ^ { b } f \right| \leq \int _ { a } ^ { b } | f |
	$$.
\end{theorem}
\begin{theorem}
	If $f$ is a piecewise continuous function or a bounded piecewise
	monotonic function on $[ a , b ] ,$ then $f$ is integrable on $[ a , b ] .$
\end{theorem}
\begin{theorem}[IVT for integrals]
	If $f$ is a continuous function on $[ a , b ] ,$ then for at least one $x$ in $( a , b )$
	we have $$
	f ( x ) = \frac { 1 } { b - a } \int _ { a } ^ { b } f
	$$
\end{theorem}
\begin{theorem}[Dominated Convergence Theorem]
	Suppose $\left( f _ { n } \right)$ is a sequence of integrable functions on $[ a , b ]$ and $f _ { n } \rightarrow$
	$f$ pointwise where $f$ is an integrable function on $[ a , b ]$ . If there exists
	an $M > 0$ such that $\left| f _ { n } ( x ) \right| \leq M$ for all $n$ and all $x$ in $[ a , b ] ,$ then $$
	\lim _ { n \rightarrow \infty } \int _ { a } ^ { b } f _ { n } ( x ) d x = \int _ { a } ^ { b } \lim _ { n \rightarrow \infty } f _ { n } ( x ) d x
	$$
\end{theorem}
\begin{corollary}[Monotone Convergence Theorem]
	Suppose $\left( f _ { n } \right)$ is a sequence of integrable functions on $[ a , b ]$ such that
	$f _ { 1 } ( x ) \leq f _ { 2 } ( x ) \leq \cdots$ for all $x$ in $[ a , b ] .$ Suppose also that $f _ { n } \rightarrow f$
	pointwise where $f$ is an integrable function on $[ a , b ] .$ Then $$
	\lim _ { n \rightarrow \infty } \int _ { a } ^ { b } f _ { n } ( x ) d x = \int _ { a } ^ { b } \lim _ { n \rightarrow \infty } f _ { n } ( x ) d x
	$$
\end{corollary}

\section{Fundamental Theorem of Calculus}

\end{document}








