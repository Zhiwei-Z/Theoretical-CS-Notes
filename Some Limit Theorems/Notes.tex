% --------------------------------------------------------------
% This is all preamble stuff that you don't have to worry about.
% Head down to where it says ``Start here"
% --------------------------------------------------------------
 
\documentclass[12pt]{article}
 
\usepackage[utf8]{inputenc}
\usepackage[english]{babel}

 
\usepackage[margin=1in]{geometry} 
\usepackage{amsmath,amsthm,amssymb}
\usepackage{graphicx}
\usepackage{tabto}

% \newtheorem{problem}{Problem}
% \newtheorem{solution}{Solution}
 
\newcommand{\N}{\mathbb{N}}
\newcommand{\Z}{\mathbb{Z}}
\newtheorem{theorem}{Theorem}[section]
\newtheorem{corollary}{Corollary}[theorem]
\newtheorem{lemma}[theorem]{Lemma}
\newtheorem{definition}[theorem]{Definition}

\newenvironment{solution}[2][Solution]{\begin{trivlist}
\item[\hskip \labelsep {\bfseries #1}\hskip \labelsep {\bfseries #2.}]}{\end{trivlist}}
 
\newenvironment{problem}[2][Problem]{\begin{trivlist}
\item[\hskip \labelsep {\bfseries #1}\hskip \labelsep {\bfseries #2.}]}{\end{trivlist}}
 
%\newenvironment{theorem}[2][Theorem]{\begin{trivlist}
%\item[\hskip \labelsep {\bfseries #1}\hskip \labelsep {\bfseries #2.}]}{\end{trivlist}}
\newenvironment{claim}[2][Claim]{\begin{trivlist}
		\item[\hskip \labelsep {\bfseries #1}\hskip \labelsep {\bfseries #2.}]}{\end{trivlist}}
\newenvironment{exercise}[2][Exercise]{\begin{trivlist}
\item[\hskip \labelsep {\bfseries #1}\hskip \labelsep {\bfseries #2.}]}{\end{trivlist}}
\newenvironment{reflection}[2][Reflection]{\begin{trivlist}
\item[\hskip \labelsep {\bfseries #1}\hskip \labelsep {\bfseries #2.}]}{\end{trivlist}}
\newenvironment{proposition}[2][Proposition]{\begin{trivlist}
\item[\hskip \labelsep {\bfseries #1}\hskip \labelsep {\bfseries #2.}]}{\end{trivlist}}



\newenvironment{example}[2][Example]{\begin{trivlist}
		\item[\hskip \labelsep {\bfseries #1}\hskip \labelsep {\bfseries #2.}]}{\end{trivlist}}
	
\newenvironment{algorithm}[2][Algorithm]{\begin{trivlist}
		\item[\hskip \labelsep {\bfseries #1}\hskip \labelsep {\bfseries #2.}]}{\end{trivlist}}
 
\begin{document}
 
% --------------------------------------------------------------
%                         Start here
% --------------------------------------------------------------
 
%\renewcommand{\qedsymbol}{\filledbox}
 
\title{Real Analysis Theorems}
\author{Zhiwei Zhang}
 
\maketitle
\section{Limsup and Liminf}
\begin{corollary}
	If $\lim \left| \frac { s _ { n + 1 } } { s _ { n } } \right|$ exists $[$ and equals $L ] ,$ then $\lim \left| s _ { n } \right| ^ { 1 / n } $ exists $[$ and
	equals $L ] .$
\end{corollary}
\section{Power Series}
Given power series $\sum _ { n = 0 } ^ { \infty } a _ { n } x ^ { n }$
\begin{theorem}
Given any $(a_n)$, one of the following holds true:
\begin{enumerate}
	\item The power series converges for all $x \in \mathbb{R}$
	\item The power series converges only for $x = 0$
	\item The power series converges for all $x$ in some bounded interval centered at $0 ;$ the intervalmay be open, half-open or closed.
\end{enumerate}
\end{theorem}

\begin{theorem}
	Let $$
	\beta = \limsup \left| a _ { n } \right| ^ { 1 / n } \quad \text { and } \quad R = \frac { 1 } { \beta }
	$$
	Then
	\begin{enumerate}
		\item The power series converges for  $|x| < R$
		\item The power series diverges for $|x| > R$
	\end{enumerate}
	
	Also notice that $\lim \left| \frac { a _ { n + 1 } } { a _ { n } } \right| = \beta$, therefore most of the time we will use $\lim \left| \frac { a _ { n + 1 } } { a _ { n } } \right|$ as it's easier to compute thatn $\beta$.
\end{theorem}
\section{More on Uniform Convergence}
\begin{theorem}
	Let $\left( f _ { n } \right)$ be a sequence of continuous functions on $[ a , b ] ,$ and suppose
	$f _ { n } \rightarrow f$ uniformly on $[ a , b ] .$ Then
	$$\lim _ { n \rightarrow \infty } \int _ { a } ^ { b } f _ { n } ( x ) d x = \int _ { a } ^ { b } f ( x ) d x$$
\end{theorem}
\begin{definition}
	A sequence $\left( f _ { n } \right)$ of functions defined on a set $S \subseteq \mathbb { R }$ is uniformly
	Cauchy on $S$ if
	\begin{center}
			$\quad$ for each $\epsilon > 0$ there exists a number $N$ such that\\
		$\quad \left| f _ { n } ( x ) - f _ { m } ( x ) \right| < \epsilon$ for all $x \in S$ and all $m , n > N$
	\end{center}

\end{definition}
\begin{theorem}
	Let $\left( f _ { n } \right)$ be a sequence of functions defined and uniformly Cauchy on
	a set $S \subseteq \mathbb { R } .$ Then there exists a function $f$ on $S$ such that $f _ { n } \rightarrow f$
	uniformly on $S .$
\end{theorem}

\end{document}








