% --------------------------------------------------------------
% This is all preamble stuff that you don't have to worry about.
% Head down to where it says ``Start here"
% --------------------------------------------------------------
 
\documentclass[12pt]{article}
 

 
\usepackage[margin=1in]{geometry} 
\usepackage{amsmath,amsthm,amssymb}
\usepackage{graphicx}
\usepackage{tabto}

% \newtheorem{problem}{Problem}
% \newtheorem{solution}{Solution}
 
\newcommand{\N}{\mathbb{N}}
\newcommand{\Z}{\mathbb{Z}}

\newenvironment{solution}[2][Solution]{\begin{trivlist}
\item[\hskip \labelsep {\bfseries #1}\hskip \labelsep {\bfseries #2.}]}{\end{trivlist}}
 
\newenvironment{problem}[2][Problem]{\begin{trivlist}
\item[\hskip \labelsep {\bfseries #1}\hskip \labelsep {\bfseries #2.}]}{\end{trivlist}}
 
\newenvironment{theorem}[2][Theorem]{\begin{trivlist}
\item[\hskip \labelsep {\bfseries #1}\hskip \labelsep {\bfseries #2.}]}{\end{trivlist}}
\newenvironment{claim}[2][Claim]{\begin{trivlist}
		\item[\hskip \labelsep {\bfseries #1}\hskip \labelsep {\bfseries #2.}]}{\end{trivlist}}
\newenvironment{lemma}[2][Lemma]{\begin{trivlist}
\item[\hskip \labelsep {\bfseries #1}\hskip \labelsep {\bfseries #2.}]}{\end{trivlist}}
\newenvironment{exercise}[2][Exercise]{\begin{trivlist}
\item[\hskip \labelsep {\bfseries #1}\hskip \labelsep {\bfseries #2.}]}{\end{trivlist}}
\newenvironment{reflection}[2][Reflection]{\begin{trivlist}
\item[\hskip \labelsep {\bfseries #1}\hskip \labelsep {\bfseries #2.}]}{\end{trivlist}}
\newenvironment{proposition}[2][Proposition]{\begin{trivlist}
\item[\hskip \labelsep {\bfseries #1}\hskip \labelsep {\bfseries #2.}]}{\end{trivlist}}
\newenvironment{corollary}[2][Corollary]{\begin{trivlist}
\item[\hskip \labelsep {\bfseries #1}\hskip \labelsep {\bfseries #2.}]}{\end{trivlist}}
\newenvironment{definition}[2][Definition]{\begin{trivlist}
		\item[\hskip \labelsep {\bfseries #1}\hskip \labelsep {\bfseries #2.}]}{\end{trivlist}}


\newenvironment{example}[2][Example]{\begin{trivlist}
		\item[\hskip \labelsep {\bfseries #1}\hskip \labelsep {\bfseries #2.}]}{\end{trivlist}}
	
\newenvironment{algorithm}[2][Algorithm]{\begin{trivlist}
		\item[\hskip \labelsep {\bfseries #1}\hskip \labelsep {\bfseries #2.}]}{\end{trivlist}}
 
\begin{document}
 
% --------------------------------------------------------------
%                         Start here
% --------------------------------------------------------------
 
%\renewcommand{\qedsymbol}{\filledbox}
 
\title{Spectral Graph Theory}
\author{Zhiwei Zhang}

 
\maketitle
\section{Introduction}
\subsection{Adjacency Matrix Representation of A Graph}
We represent a Graph $G$ with a square adjacency matrix $A$, where entry $$a_{ij} = \begin{cases}
1 &\text{if (i, j) is an edge in }G \\
0 &\text{otherwise}
\end{cases}$$
Evidently, the matrix is symmetric, or $a_{ij} = a_{ji}$. Therefore by spectral theorem, we have a orthonormal basis of eigenvectors, with real eigenvalues associated.

\section{Counting Paths with Adjacency Matrix}
\begin{theorem}
	Let $G$ be a graph on labeled vertices, let $A$ be its adjacency
	matrix, and let $k$ be a positive integer. Then $A_{i,j}^{k}$ is equal to the number of
	walks from $i$ to $j$ that are of length $k$ .
\end{theorem} 

\begin{proof}
	The proof is fairly simple, and we will do it by induction.\\
	\newline
	When $k = 1$, we look at the original adjacency matrix, and $A_{i, j}$ indicates whether there's an edge between $i, j$, which is a path of length 1.\\
	\newline
	Now assume that the statement is true for $k$, and
	prove it for $k + 1$. \\
	\newline
	\textbf{Let's first think about it intuitively, $A^{k}$ gives the number of paths walks from $i$ to all other points. 
		If one such point is $v$, then we just need to determine if there's and edge from $v$ to $j$, 
		if so then we just add the number of walks from $i$ to  $k$.}\\
	\newline
	Let $z$ be any vertex of $G$. If there are $b_{i,z}$ walks of length $k$ from $i$ to $z$, and there are $a_{z,j}$ walks of length one (in other words, edges)
	from $z$ to $j$, then there are $b_{i,z}a_{z,j}$ walks of length $k + 1$ from $i$ to $j$ whose
	next-to-last vertex is $z$. Therefore, the number of all walks of length $k + 1$ from $i$ to $j$ is:
	$$
	c(i,j) = \sum_{z \in G} b_{i,z}a_{z,j}
	$$
	Since $b_{i, z}$ correspond to an entry in $A^k$, the formula above is basically a matrix multiplication. 
\end{proof}

At this point, it's a good habit to check our proof again. Ask ourselves this question:
\begin{center}
	\textit{We know $A^{k}_{i, j}$ represents some number of walks from $i$ to $j$, but does it count all of them?}
\end{center}
We'll leave it as a quick thought exercise.
\subsection{Connectivity}
\begin{theorem}
	Let $G$ be a simple graph on $n$ vertices, and let $A$ be the
	adjacency matrix of $G .$ Then $G$ is connected iff $( I + A ) ^ { n - 1 }$
	consists of strictly positive entries.
\end{theorem}
\begin{proof}
	We will only give the central idea of the proof here\\
	\newline
	A path from one point to another consists at most $n$ vertices, or $n-1$ edges. Therefore, if we cannot find a path between two points within $n-1$ edges, the graph is not connected\\
	\newline
	Using the previous theorem, we know that $A_{n-1}$ gives the number of paths if length $n-1$ between any two points, however, this is \textbf{not enough}.\\
	\newline
	Notice that the theorem indicates it's $(I + A)^{n-1}$ instead of $A^{n-1}$. So what does the $I$ do?\\
	\newline
	Graphically, it means that we assume any vertex is connected to itself. What is $(I + A)^{n-1}$ then? \\
	\newline
	For simplicity, call $(I + A) = A'$. With $A'_{i, i} = 1$, $A'^{k}$ no longer counts the number of paths of length \textbf{strictly} $k$, but the number of paths of length $\leq k$. This is because we don't have to move to another point every time we multiply $A'$, we can choose to stay at the same point since $A'_{i, i} = 1$.\\
	\newline
	Therefore $(I + A)^{n-1}$ counts the number of paths of length $\leq n - 1$, and if there still exist an 0 entry $A'^{n-1}_{i, j}$, then it means if we cannot find a path between two points within $n-1$ edges, the graph is not connected.
\end{proof}
\end{document}








