% --------------------------------------------------------------
% This is all preamble stuff that you don't have to worry about.
% Head down to where it says ``Start here"
% --------------------------------------------------------------
 
\documentclass[12pt]{article}
 

 
\usepackage[margin=1in]{geometry} 
\usepackage{amsmath,amsthm,amssymb}
\usepackage{graphicx}
\usepackage{tabto}

% \newtheorem{problem}{Problem}
% \newtheorem{solution}{Solution}
 
\newcommand{\N}{\mathbb{N}}
\newcommand{\Z}{\mathbb{Z}}

\newenvironment{solution}[2][Solution]{\begin{trivlist}
\item[\hskip \labelsep {\bfseries #1}\hskip \labelsep {\bfseries #2.}]}{\end{trivlist}}
 
\newenvironment{problem}[2][Problem]{\begin{trivlist}
\item[\hskip \labelsep {\bfseries #1}\hskip \labelsep {\bfseries #2.}]}{\end{trivlist}}
 
\newenvironment{theorem}[2][Theorem]{\begin{trivlist}
\item[\hskip \labelsep {\bfseries #1}\hskip \labelsep {\bfseries #2.}]}{\end{trivlist}}
\newenvironment{claim}[2][Claim]{\begin{trivlist}
		\item[\hskip \labelsep {\bfseries #1}\hskip \labelsep {\bfseries #2.}]}{\end{trivlist}}
\newenvironment{lemma}[2][Lemma]{\begin{trivlist}
\item[\hskip \labelsep {\bfseries #1}\hskip \labelsep {\bfseries #2.}]}{\end{trivlist}}
\newenvironment{exercise}[2][Exercise]{\begin{trivlist}
\item[\hskip \labelsep {\bfseries #1}\hskip \labelsep {\bfseries #2.}]}{\end{trivlist}}
\newenvironment{reflection}[2][Reflection]{\begin{trivlist}
\item[\hskip \labelsep {\bfseries #1}\hskip \labelsep {\bfseries #2.}]}{\end{trivlist}}
\newenvironment{proposition}[2][Proposition]{\begin{trivlist}
\item[\hskip \labelsep {\bfseries #1}\hskip \labelsep {\bfseries #2.}]}{\end{trivlist}}
\newenvironment{corollary}[2][Corollary]{\begin{trivlist}
\item[\hskip \labelsep {\bfseries #1}\hskip \labelsep {\bfseries #2.}]}{\end{trivlist}}
\newenvironment{definition}[2][Definition]{\begin{trivlist}
		\item[\hskip \labelsep {\bfseries #1}\hskip \labelsep {\bfseries #2.}]}{\end{trivlist}}


\newenvironment{example}[2][Example]{\begin{trivlist}
		\item[\hskip \labelsep {\bfseries #1}\hskip \labelsep {\bfseries #2.}]}{\end{trivlist}}
	
\newenvironment{algorithm}[2][Algorithm]{\begin{trivlist}
		\item[\hskip \labelsep {\bfseries #1}\hskip \labelsep {\bfseries #2.}]}{\end{trivlist}}
 
\begin{document}
 
% --------------------------------------------------------------
%                         Start here
% --------------------------------------------------------------
 
%\renewcommand{\qedsymbol}{\filledbox}
 
\title{Spectral Graph Theory -- Electric Flow}
\author{Zhiwei Zhang}

 
\maketitle
\section{Introduction}
In this notes, we are going to discuss the interestingproperties when turning a graph into a network of resistors.

\subsection{Electrical Laws}
First recall some E\&M Laws:
\begin{align*}
&I = \frac{U}{R} & \text{ Ohm's law}\\
&E = I^2R &\text{ Energy formula}\\
&|I_{v, in}| = |I_{v, out}| & \text{ conservation of flow}\\
\end{align*}
Note that the last law only holds for nodes that are not source or sink.

\subsection{Formation on Graph}
We will write a matrix formation of the problem.\\
\begin{itemize}
	\item Let $G(V, E)$ be an undirected graph with $|V| = n, |E| = m$.
	\item Let $v \in \mathbb{R}^n$ be the vector representing the potentials of vertices.
	\item Edges represent the resistors, and $\forall e(u, v) \in E$. Edge $e$ has resistance $r_e$.
	\item Let $f \in \mathbb{R}^m$ representing the flow of all edges, where $f(a, b)$ represents the flow from $a$ to $b$ with. Since $f(a, b)$ is directed, we have $f(a, b) = -f(b, a)$.
	\item Let weight $w_e= \frac{1}{r_e}$, or the "conductance" of $e$.
\end{itemize}

\section{Matrix Formation}
We also define $ f _ { e x t } ( a ) = \sum _ { b = ( a , b ) \in E } f ( a , b )$, and $f_{ex}t(a)$ basically denotes the external current on $a$, which is \textbf{positive number if $a$ is souce, negative number with equal magnitude if $a$ is source, and zero otherwise}. So $f_{ext}$ is a very sparse vector.\\
\newline
Ohm's law directly states that $f ( a , b ) = \frac { v ( a ) - v ( b ) } { r _ { a , b } } = w _ { a , b } ( v ( a ) - v ( b ) )$, therefore $$\sum _ { b : ( a , b ) \in E } f ( a , b ) = \sum _ { b : ( a , b ) \in E } w _ { a , b } ( v ( a ) - v ( b ) ) = d ( a ) v ( a ) - \sum _ { b : ( a , b ) \in E } w _ { a, b  } v ( b )$$ where $d(a) = \sum_{b : ( a , b ) \in E }w_{a, b}$, the weighted degree of $a$.\\
\newline
Notice that $d(a), w_{a, b}$ are entries of the weighted laplacian $L_G$, and through simple verification, we can show that the equation above the equivalent to $L_Gv = f_{ext}$


\end{document}








