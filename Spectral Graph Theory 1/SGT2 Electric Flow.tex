% --------------------------------------------------------------
% This is all preamble stuff that you don't have to worry about.
% Head down to where it says ``Start here"
% --------------------------------------------------------------
 
\documentclass[12pt]{article}
 

 
\usepackage[margin=1in]{geometry} 
\usepackage{amsmath,amsthm,amssymb}
\usepackage{graphicx}
\usepackage{tabto}

% \newtheorem{problem}{Problem}
% \newtheorem{solution}{Solution}
 
\newcommand{\N}{\mathbb{N}}
\newcommand{\Z}{\mathbb{Z}}

\newenvironment{solution}[2][Solution]{\begin{trivlist}
\item[\hskip \labelsep {\bfseries #1}\hskip \labelsep {\bfseries #2.}]}{\end{trivlist}}
 
\newenvironment{problem}[2][Problem]{\begin{trivlist}
\item[\hskip \labelsep {\bfseries #1}\hskip \labelsep {\bfseries #2.}]}{\end{trivlist}}
 
\newenvironment{theorem}[2][Theorem]{\begin{trivlist}
\item[\hskip \labelsep {\bfseries #1}\hskip \labelsep {\bfseries #2.}]}{\end{trivlist}}
\newenvironment{claim}[2][Claim]{\begin{trivlist}
		\item[\hskip \labelsep {\bfseries #1}\hskip \labelsep {\bfseries #2.}]}{\end{trivlist}}
\newenvironment{lemma}[2][Lemma]{\begin{trivlist}
\item[\hskip \labelsep {\bfseries #1}\hskip \labelsep {\bfseries #2.}]}{\end{trivlist}}
\newenvironment{exercise}[2][Exercise]{\begin{trivlist}
\item[\hskip \labelsep {\bfseries #1}\hskip \labelsep {\bfseries #2.}]}{\end{trivlist}}
\newenvironment{reflection}[2][Reflection]{\begin{trivlist}
\item[\hskip \labelsep {\bfseries #1}\hskip \labelsep {\bfseries #2.}]}{\end{trivlist}}
\newenvironment{proposition}[2][Proposition]{\begin{trivlist}
\item[\hskip \labelsep {\bfseries #1}\hskip \labelsep {\bfseries #2.}]}{\end{trivlist}}
\newenvironment{corollary}[2][Corollary]{\begin{trivlist}
\item[\hskip \labelsep {\bfseries #1}\hskip \labelsep {\bfseries #2.}]}{\end{trivlist}}
\newenvironment{definition}[2][Definition]{\begin{trivlist}
		\item[\hskip \labelsep {\bfseries #1}\hskip \labelsep {\bfseries #2.}]}{\end{trivlist}}


\newenvironment{example}[2][Example]{\begin{trivlist}
		\item[\hskip \labelsep {\bfseries #1}\hskip \labelsep {\bfseries #2.}]}{\end{trivlist}}
	
\newenvironment{algorithm}[2][Algorithm]{\begin{trivlist}
		\item[\hskip \labelsep {\bfseries #1}\hskip \labelsep {\bfseries #2.}]}{\end{trivlist}}
 
\begin{document}
 
% --------------------------------------------------------------
%                         Start here
% --------------------------------------------------------------
 
%\renewcommand{\qedsymbol}{\filledbox}
 
\title{Spectral Graph Theory -- Electric Flow}
\author{Zhiwei Zhang}

 
\maketitle
\section{Introduction}
In this notes, we are going to discuss the interestingproperties when turning a graph into a network of resistors.

\subsection{Electrical Laws}
First recall some E\&M Laws:
\begin{align*}
&I = \frac{U}{R} & \text{ Ohm's law}\\
&E = I^2R &\text{ Energy formula}\\
&|I_{v, in}| = |I_{v, out}| & \text{ conservation of flow}\\
\end{align*}
Note that the last law only holds for nodes that are not source or sink.

\subsection{Matrices}
Also recall the laplacian of a graph:
$$L _ { i , j } : = \left\{ \begin{array} { l l } { \operatorname { deg } \left( v _ { i } \right) } & { \text { if } i = j } \\ { - 1 } & { \text { if } i \neq j \text { and } v _ { i } \text { is adjacent to } v _ { j } } \\ { 0 } & { \text { otherwise } } \end{array} \right.$$
Weighted laplacian of a graph:
$$
L _ { i j } = \left\{ \begin{array} { l l } { - w _ { i j } } & { \text { if } i \sim j } \\ { w _ { i } } & { \text { if } i = j } \\ { 0 } & { \text { otherwise } } \end{array} \right.
$$
where $w _ { i } = \sum _ { j \sim i } w _ { i j }$ is the sum of the weights of edges incident on vertex $i$.\\
\newline
Pseudoinverse of laplacian:
Let $0 = \lambda _ { 1 } < \lambda _ { 2 } \leqslant \ldots \leq \lambda _ { n }$ be the eigenvalues of  $L_G$ with associated eigenvectors $u _ { 1 } , u _ { 2 } , \ldots , u _ { n }$. Then $$L _ { G } = \sum _ { i = 1 } ^ { n } \lambda _ { i } u _ { i } u _ { i } ^ { T }$$. The pseudo-inverse of $L_G$ is $$L _ { G } ^ { + } = \sum _ { i = 2 } ^ { n } \frac { 1 } { \lambda _ { i } } u _ { i } n _ { i } ^ { T }$$.

\subsection{Formation on Graph}
We will write a matrix formation of the problem.\\
\begin{itemize}
	\item Let $G(V, E)$ be an undirected graph with $|V| = n, |E| = m$.
	\item Let $v \in \mathbb{R}^n$ be the vector representing the potentials of vertices.
	\item Edges represent the resistors, and $\forall e(u, v) \in E$. Edge $e$ has resistance $r_e$.
	\item Let $f \in \mathbb{R}^m$ representing the flow of all edges, where $f(a, b)$ represents the flow from $a$ to $b$ with. Since $f(a, b)$ is directed, we have $f(a, b) = -f(b, a)$.
	\item Let weight $w_e= \frac{1}{r_e}$, or the "conductance" of $e$.
\end{itemize}

\section{Matrix Formation}
We also define $ f _ { e x t } ( a ) = \sum _ { b = ( a , b ) \in E } f ( a , b )$, and $f_{ex}t(a)$ basically denotes the external current on $a$, which is \textbf{positive number if $a$ is souce, negative number with equal magnitude if $a$ is source, and zero otherwise}. So $f_{ext}$ is a very sparse vector.\\
\newline
Ohm's law directly states that $f ( a , b ) = \frac { v ( a ) - v ( b ) } { r _ { a , b } } = w _ { a , b } ( v ( a ) - v ( b ) )$, therefore $$\sum _ { b : ( a , b ) \in E } f ( a , b ) = \sum _ { b : ( a , b ) \in E } w _ { a , b } ( v ( a ) - v ( b ) ) = d ( a ) v ( a ) - \sum _ { b : ( a , b ) \in E } w _ { a, b  } v ( b )$$ where $d(a) = \sum_{b : ( a , b ) \in E }w_{a, b}$, the weighted degree of $a$.\\
\newline
Notice that $d(a), w_{a, b}$ are entries of the weighted laplacian $L_G$, and through simple verification, we can show that the equation above the equivalent to $L_Gv = f_{ext}$

\section{Computing Voltages}
Since we know that $Nul(L_G) = \vec{1}$, it's trivial to see that for any vector $x$, $L_G x$ is perpendicular to $\vec{1}$, which implies there's a solution to $L_Gv = f_{ext}$ iff $f_{ext}$ is perpendicular to 1. This is also simply true since the two non-zero entries of $f_{ext}$ have the same magnitude with opposite sign.\\
\newline
Therefore $v = L_{G}^+ f_{ext}$ is the only solution with $v \perp \vec{1}$, and the whole set of solution is $\{v + c\vec{1} | c\in \mathbb{R}\}$ \\
\newline
This also makes sense in the physical way, as if we increase the potential of all nodes, the physical flow or energy will not change as electrical potentials are only significant when taking differences.

\section{Computing Currents}
First we reintroduce the incidence matrix $B$ of dimension $m \times n$, and the row corresponding to the edge $e = (a, b)$ where $a < b$ is $(x_a - x_b)^T$, where $x_a$ is the characteristic with the only non-zero entry at $a-th$ entry of value 1.\\
\newline
Let $w$ be the mxm diagonal matrix where $W_{e, e} = w_e$ is the weight of edge $e$.\\
\newline
Notice that $Bv$ gives the potential difference on each edge, and therefore $f = WBv$.\\
\newline
It's also true that $L _ { G } = \sum _ { e : ( a , b ) } w _ { e } \left( x _ { a } - x _ { b } \right) \left( x _ { a } - x _ { b } \right) ^ { T } = B ^ { T } W B$. \\
\newline Therefore $f _ { e x t } = L _ { G } v = B ^ { T } W B v = B ^ { T } f$

\newpage

\section{Random Walk and Effective Resistance}
Random walk is largely studied on graphs, and it turns out that the concept of effective resistance is very useful in this area. Before we get into the theorems, we need to specify some restrictions on the graph we are using and define some terms.
\subsection{Specification and Definitions}
When modeling random walks with electric flows, we general set the resistance of each edge to be 1, and $\forall e \in E, w_e = w/r_e = 1$. \\
\newline
In addition:
\begin{itemize}
	\item We define hitting time $h_{uv} := $ expected number of steps to reach $v$ from $u$
	\item We define commutime $C_{u, v} := $ expected number of steps to start from $u$, travelling to $v$, and come back to $u$. More specifically, $C_{u, v} = h_{u, v} + h_{v, u} = C_{v, u}$. Notice that $h_{u, v}$ not necessarily = $ h_{v, u}$.
\end{itemize}

\subsection{Theorems and Corollaries}
\begin{theorem}{1}
	$C_{s, t} = 2mR_{eff}(s, t)$
\end{theorem}
\begin{proof}
	First using recursive definition, we can develop the equation: $$\forall v \in V - t, h_{vt} = \sum_{w : vw \in E}\frac{1}{d(v)}(1 + h_{wt}).$$
	Notice that this only holds for all vertices except $t$, which is very important since $h_{tt} = 0$, and there'd also be no reason to walk away from $t$ and walk back. We will reemphasize this point later again.\\
	\newline
	The equation is equivalent to $$d(v) = d(v)h_{vt} - \sum_{w : vw \in E}h_{wt} = \sum _ {  w : vw \in E} \left( h _ { v t } - h _ { w t } \right)$$
	Notice that this form is similar to the Laplacian system of linear equations (reacall $\sum _ { b : ( a , b ) \in E } f ( a , b ) = \sum _ { b : ( a , b ) \in E } w _ { a , b } ( v ( a ) - v ( b ) ) = d ( a ) v ( a ) - \sum _ { b : ( a , b ) \in E } w _ { a , b } v ( b )$)\\
	\newline
	Next, we denote $\phi_{vt}$ be the potential at $v$ with $\phi_{tt = 0}$. Now if for all $v \in V - t$, we inject $d(v)$ units of current (so $|V| - 1$ sources). Totally we inject $2m - d(t)$ units of current, and we have to remove the same amount from $t$. \\
	\newline
	Since $\forall v \in V - t$, $d(v)$ units of current are injected, its external current is therefore $d(v)$, which dissipates through its neighbors. Therefore $$d(v) = \sum_{w : vw\in E}(\phi_{vt} - \phi_{wt})/r_{(v, w)} = \sum_{w : vw\in E}(\phi_{vt} - \phi_{wt}).$$ Let $f_{t}$ be the external flow vector if $t$ is the only sink. In this situation, $\forall v \in V-t, f_{t}(v) = d(v, f_{t}(t) = d(t) - 2m)$\\
	\newline
	Therefore we get $\forall v \in V-t, d(v) = \sum_{w : vw\in E}(\phi_{vt} - \phi_{wt}) = \sum _ {  w : vw \in E} \left( h _ { v t } - h _ { w t } \right)$, so $\vec{\phi_t}$ and $\vec{h_t}$ satisfy the same equation: $L_{G}x  = f_{t}$. Since we have a set of possible solutions, but also since $h_{tt} = \phi_{tt} = 0$, $\vec{h_t} = \vec{\phi_t}$ at all points. We now reamphasize the importance of $d(v) = \sum _ {  w : vw \in E} \left( h _ { v t } - h _ { w t } \right)$ only satisfies for $v \in V - t$: we can see that if we extend this definition to $t$, they will not satisfy the laplacian equation.\\
	\newline
	Now, we define $f_s$ similar to $f_t$ (now $s$ is the only sink, all other vertices are sources).\\
	\newline
	Therefore we have $L _ { G } \left( \vec { h } _ { t } - h _ { S } \right) = f _ { t } - f _ { S } = 2 m \left( x _ { s } - x _ { t } \right)$, and so $\left( \vec { h _ { t } } - \vec { h } _ { s } \right) / 2 m = L _ { G } ^ { \dagger } \left( x _ { s } - x _ { t } \right)$. 
	\begin{align*}
	R _ { e f f } ( s , t ) &=( x _ { s } - x _ { t })^T L{G}^{+}( x _ { s } - x _ { t }) \\
							&= \left( x _ { s } - x _ { t } \right) ^ { T } \left( \frac { 1 } { 2 m } \left( \vec { h } _ { t } - \vec { h } _ { s } \right) \right) \\
							&= \frac { 1 } { 2 m } \left( h _ { t } ( s ) + \vec { h } _ { s } ( t ) \right) \\
							&= \frac { 1 } { 2 m } \left( h _ { s t } + h _ { t s } \right)\\
							&= \frac { 1 } { 2 m } C_{s, t}
	\end{align*}
\end{proof}

\begin{theorem}{2}
	The cover time of an undirected graph is at most $2m(n - 1)$.
\end{theorem}
\begin{proof}
	Let $T$ be a spanning tree of $G$.\\
	\newline
	Consider this tree traversal that cover all vertices: follow DFS. Basically if we encounter branching, we will choose one branch and go deeper and return back and go through the next branch, which cover all the vertices (each edge is visited twice).\\
	\newline
	Therefore the cover time of $G$ is bounded by the expected length of the walk, which is at most $\sum_{uv \in T}(h_{uv} + h_{vu}) = \sum_{uv \in T}C_{uv} \leq 2m(n-1)$ since the effective resistance between two vertices is at most 1.
\end{proof}

The theorem gives an upper bound of $2m(n - 1) \leq (n)(n - 1)^2 < n^3$, but we can get tighter bounds.

\begin{theorem}{3}
	Let $R ( G ) = \max _ { u , v } R _ { e f f } ( u , v ) \quad$ Then $\quad m R ( G ) \leqslant$ cover tine $\leqslant 2 e ^ { 3 } m R ( n ) \log n + n$
\end{theorem}
\begin{proof}
	\begin{enumerate}
		\item The lower bound is easy to derive: Let $R ( G ) = R _ { e f f } ( u , v ) \cdot$ Then $2 m R _ { e f f } ( u , \omega ) = C _ { u v } = h _ { u v } + h _ { v n }$, so the cover time is at least $\max \left\{ h _ { u v } , h _ { v u } \right\} \geq C _ { n v } / 2 = m R _ { e f f } ( u , v )$
		\item To compute the upper bound, we first observe that the hitting time between any pair of vertices is $2mR(G)$ (the max commute time). Therefore the probability of  after $2me^3R(G)\ln{n}$ steps and some vertex is still not hit is at most $\frac{1}{e^3 \cdot \ln{n}}$ by Markov inequality. We can also observe that $\frac{1}{e^3 \cdot \ln{n}} \leq 1/n^3$ when $n \geq 3$.\\
		\newline
		Therefore by union bound, the probability of some vertex is not hit is at most $\frac{1}{n^2}$. In this situation, we will use the "bad" bound of $n^3$. Therefore the cover time is at most $2 e ^ { 3 } m R ( G ) \ln n + \left( \frac { 1 } { n ^ { 2 } } \right) n ^ { 3 } = 2 e ^ { 3 } m R ( 4 ) \ln n + n$.
	\end{enumerate}
\end{proof}

\end{document}








