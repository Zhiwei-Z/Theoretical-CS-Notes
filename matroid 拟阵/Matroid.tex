% --------------------------------------------------------------
% This is all preamble stuff that you don't have to worry about.
% Head down to where it says ``Start here"
% --------------------------------------------------------------

\documentclass[12pt]{article}



\usepackage[margin=1in]{geometry} 
\usepackage{amsmath,amsthm,amssymb}
\usepackage{graphicx}
\usepackage{tabto}

% \newtheorem{problem}{Problem}
% \newtheorem{solution}{Solution}

\newcommand{\N}{\mathbb{N}}
\newcommand{\Z}{\mathbb{Z}}

\newenvironment{solution}[2][Solution]{\begin{trivlist}
		\item[\hskip \labelsep {\bfseries #1}\hskip \labelsep {\bfseries #2.}]}{\end{trivlist}}

\newenvironment{problem}[2][Problem]{\begin{trivlist}
		\item[\hskip \labelsep {\bfseries #1}\hskip \labelsep {\bfseries #2.}]}{\end{trivlist}}

\newenvironment{theorem}[2][Theorem]{\begin{trivlist}
		\item[\hskip \labelsep {\bfseries #1}\hskip \labelsep {\bfseries #2.}]}{\end{trivlist}}
\newenvironment{claim}[2][Claim]{\begin{trivlist}
		\item[\hskip \labelsep {\bfseries #1}\hskip \labelsep {\bfseries #2.}]}{\end{trivlist}}
\newenvironment{lemma}[2][Lemma]{\begin{trivlist}
		\item[\hskip \labelsep {\bfseries #1}\hskip \labelsep {\bfseries #2.}]}{\end{trivlist}}
\newenvironment{exercise}[2][Exercise]{\begin{trivlist}
		\item[\hskip \labelsep {\bfseries #1}\hskip \labelsep {\bfseries #2.}]}{\end{trivlist}}
\newenvironment{reflection}[2][Reflection]{\begin{trivlist}
		\item[\hskip \labelsep {\bfseries #1}\hskip \labelsep {\bfseries #2.}]}{\end{trivlist}}
\newenvironment{proposition}[2][Proposition]{\begin{trivlist}
		\item[\hskip \labelsep {\bfseries #1}\hskip \labelsep {\bfseries #2.}]}{\end{trivlist}}
\newenvironment{corollary}[2][Corollary]{\begin{trivlist}
		\item[\hskip \labelsep {\bfseries #1}\hskip \labelsep {\bfseries #2.}]}{\end{trivlist}}
\newenvironment{definition}[2][Definition]{\begin{trivlist}
		\item[\hskip \labelsep {\bfseries #1}\hskip \labelsep {\bfseries #2.}]}{\end{trivlist}}


\newenvironment{example}[2][Example]{\begin{trivlist}
		\item[\hskip \labelsep {\bfseries #1}\hskip \labelsep {\bfseries #2.}]}{\end{trivlist}}

\newenvironment{algorithm}[2][Algorithm]{\begin{trivlist}
		\item[\hskip \labelsep {\bfseries #1}\hskip \labelsep {\bfseries #2.}]}{\end{trivlist}}

\begin{document}
	
	% --------------------------------------------------------------
	%                         Start here
	% --------------------------------------------------------------
	
	%\renewcommand{\qedsymbol}{\filledbox}
	
	\title{Matroid}
	\author{Zhiwei Zhang}
	
	\maketitle
	
	\section{Matriod}
	Matriod $M = (S, l)$ is an ordered pair such that: 
	\begin{enumerate}
		\item $S$ is non-empty and finite
		\item $l \subseteq 2^{S}$ ($l$ is a set containing subsets of $S$), and $\emptyset \in l$
		\item \textbf{Downward Closure}: If $B \in l$, $A \in B$, then $A \in l$. Call $B$ or any its subset an \textbf{"independent set"}.
		\item \textbf{Augmentation}: If $A \in l, B \in l,$ and $|A| < |B|$, then $\exists x \in (B \backslash A)$ such that $A \cup {x} \in l$
	\end{enumerate}
	
	\section{Matriod in Graph}
	For an undirected graph $G = (V, E)$, define Matroid $M_G = (S_G, l_G)$ as the following:
	\begin{enumerate}
		\item $S_G = E$, or the edge set of the graph
		\item If $A$ is a subset of $E$, $A \in l_C$ if and only if $A$ doesn't contain circles. Or equivalently, a collection of edges $A$ is an independent iff $G(V, A)$ creates a \textbf{forest}.
	\end{enumerate}
	Now we prove $M_G$ satisfies all conditions of a matroid:
	\begin{enumerate}
		\item $S_G$ is obviously non-empty and finite
		\item \textbf{Downward Closure (Inheritance)}: It's trivial that removing edges from a forest will also create a forest
		\item \textbf{Augmentation (Exchange Property)}: It's easy to prove that \textbf{a forest with $k$ edges have $|V| - k$ trees (counting any isolated vertex also as a tree)}. Next, if $G(V, A), G(V, B)$ are two forests of $G$, and $|A| < |B|$. Then $G(V, A)$ has $|V| - |A|$ trees, more than the trees in $G(V, B)$ which is $|V| - |B|$. Since $G(V, B)$ has less trees, there must be an edge $e \in B \backslash A$ that connects 2 trees (can also be vertices) of $A$. Since adding an edge between two trees will not create cycles, $A \cup \{e\} \in l_G$.
	\end{enumerate}
	\begin{definition}{1}
		If there exists an element $x$ not in an independent set $A$, and $A \cup {x} \in l$, we call $A$ an \textbf{extendable} independent subset, and $x$ an \textbf{extension} to $A$. 
	\end{definition}
	\begin{definition}{2}
		If $A$ doesn't have an extension, we call $A$ a \textbf{maximal} independent subset.
	\end{definition}
	
	Notice that because of the augmentation property, we can derive the following theorem:
	
	\begin{theorem}
		All maximal independet subsets have the same size.
	\end{theorem}
	The proof is trivial: or else the larger independent subsets can give extension elements to  smaller independet subsets.
	
	\section{Weighted Matroid}
	A matroid $M(S, l)$ is weighted if it's associated with a weight function $w(\cdot): S \to \mathbb{R}^{+}$. The independent subset with the maximum weight is called the \textbf{\textit{optimal}} independent subset of the matroid. \\
	\newline
	Since all the weights are positive, the \textbf{\textit{optimal}} independent subset is also a \textbf{\textit{maximal}} independent subset.
	\subsection{Matroid Greedy Algorithm}
	\textbf{Greedy(M, w):}
	\begin{enumerate}
		\item $A = \emptyset$
		\item sort $M . S$ into monotonically decreasing order by weight $w$
		\item for each $x \in M . S ,$ taken in monotonically decreasing order by weight $w ( x )$
		\item \quad if $A \cup \{ x \} \in M . \mathcal { I }$
		\item \qquad $A = A \cup \{ x \}$
		\item return $A$.
	\end{enumerate}
	Weighted matroid has the property that Matroids exhibit the greedy-choice property.
	\begin{lemma}{1}
		Suppose that $M = (S, l)$ is a weighted matroid with weight function $w$ and that $S$
		is sorted into monotonically decreasing order by weight. Let $x$ be the first element
		of $S$ such that $\{x\}$ is independent, if any such $x$ exists. If $x$ exists, then there exists
		an optimal subset $A$ of $S$ that contains $x$.
	\end{lemma}
	\begin{proof}
		If such element doesn't exist, then $l$ obviously just contains the empty set.\\
		\newline
		If such element $x$ exist, we prove by contradiction, supposing $B$ is the optimal independent subset. Construct independent subset $A$ from $\{x\}$ and keep adding elements from $B$ using augmentation to make them the same size. Therefore $A = B - {y} + {x}$ and  $w(y) < w(x)$. Therefore $A$ is obviously more weighted than $B$.
	\end{proof}
	
	\begin{lemma}{2}
		Let $M$ be any matroid. If $x$ is an element of $S$ that is an extension of some independent subset $A$ of $S$, then $x$ is also an extension of $\emptyset$.
	\end{lemma}
	\begin{proof}
		By downward closure or heridity, $\{x\}$ is a subset of $A$. Therefore $\{x\}$ is a valid independent subset of the matroid.
	\end{proof}
	
	\begin{lemma}{3}
		Let $M$ be any matroid. If $\{x\}$ is not independent, then $x$ is not and extension to any independent subset.
	\end{lemma}
	\begin{proof}
		Lemma 2's contrapositive.
	\end{proof}
	Now we can prove that:
	\begin{lemma}{4}
		The greedy algorithm produces an optimal independent subset.
	\end{lemma}
	\begin{proof}
		Let $x$ be the first element of $S$ chosen by GREEDY for the weighted matroid
		$M = ( S , l ) .$ The remaining problem of finding a maximum-weight indepen-
		dent subset containing $x$ reduces to finding a maximum-weight independent subset
		of the weighted matroid $M ^ { \prime } = \left( S ^ { \prime } , l ^ { \prime } \right) ,$ where
		$$\begin{array} { l } { S ^ { \prime } = \{ y \in S : \{ x , y \} \in l \} } \\ { l ^ { \prime } = \{ B \subseteq S - \{ x \} : B \cup \{ x \} \in l \} } \end{array}$$
		and the weight function for $M ^ { \prime }$ is the weight function for $M ,$ restricted to $S ^ { \prime } .$ (We call $M ^ { \prime }$ the \textbf{contraction} of $M$ by the element $x$ .)\\
		\newline
		In other words, there's a bijection between optimal independet sets of $M$ containing $x$ and the optimal independent sets of $M'$.
	\end{proof}


\medskip
\begin{thebibliography}{9}
	\bibitem{algBook}
	 Thomas H. Cormen, Charles E. Leiserson, Ronald L. Rivest and Clifford Stein 
	 \textit{Introduction to Algorithms, Third Edition}. The MIT Press
	 Cambridge, Massachusetts London, England. 2009
\end{thebibliography}
\end{document}



